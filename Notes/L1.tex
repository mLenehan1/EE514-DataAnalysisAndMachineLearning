\documentclass[a4paper]{article}

\usepackage[utf8]{inputenc}
\usepackage[T1]{fontenc}
\usepackage{textcomp}
\usepackage{amsmath, amssymb}


% figure support
\usepackage{import}
\usepackage{xifthen}
\pdfminorversion=7
\usepackage{pdfpages}
\usepackage{transparent}
\newcommand{\incfig}[1]{%
	\def\svgwidth{\columnwidth}
	\import{./figures/}{#1.pdf_tex}
}
\newcommand{\R}{\mathbb{R}}
\newcommand{\Z}{\mathbb{Z}}
\newcommand{\N}{\mathbb{N}}
\newcommand{\Q}{\mathbb{Q}}

\pdfsuppresswarningpagegroup=1

\begin{document}
	\section{Mathematics Refresher for Machine Learning}
	\subsection{Sets}
	A set is a well-defined collection of distinct objects (possibly infinite or uncountable).
	E.g:
	\begin{itemize}
		\item $\{1,2,3\}, \{a,e,i,o,u\}, \{\pi,e\}  $
		\item Integers $\Z = \{\ldots,-3,-2,-1,0,1,2,3,\ldots\}$
		\item Positive Integers $\Z_{++} = \{1,2,3,\ldots\} $
		\item Real Numbers $\R$
	\end{itemize}
	If x is an elements of set Z we write $x\inZ$.
	E.g: $x\in\R$ means x is a real number.
	Set builder notation:
	\begin{itemize}
		\item Positive reals $\R_{++} = \{x\in\R: x>0\} $
	\end{itemize}
	\subsection{Sets: empty set, cardinality, intersection, union}
	The empty set is the set with no elements $\0 = \{\} $
	The cardinality of a set is the number of elements in the set.
	E.g: $X = \{1,2,3\}, |X|=\#X=3$ 
	The intersection of two sets is the set containing all common elements.
	If $A = \{1,2,3\}$ and $B = \{3,4,5\}$ then the intersection $A\cap B = \{3\} $
	\[
		\Z \cap \R = \Z
	.\]
	The union of two sets is the set containing all elements that occur in either set.
	If $A = \{1,2,3\} $ and $B = \{3,4,5\} $, then the union $A \cup B = \{1,2,3,4,5\} $ 
	\[
		\Z \cup \R = \R
	.\]
	\subsection{Sets: subsets}
	A is a subset of B if all the elements of A are also contained in B.
	Written as $A \subset B$ 
	\[
		\{1,2\} \subset \{1,2,3\}
	.\]
	\[
		\Z \subset \R
	.\]
	\[
		\Z_{++} \subset \Z_+ \subset \Z \subset \R
	.\]
	\[
		\N \subset \Z \subset \Q \subset \R
	.\]
	\subsection{Vectors}
	Vectors, in general, an abstract mathematical notation, but for the purpose of this module can be
	thought of as an ordered list of numbers
	E.g: Column vectors:
	\[
		x = \begin{pmatrix} 1\\ 3\\ \end{pmatrix} 
		y = \begin{pmatrix} 3\\ 1\\ 8 \end{pmatrix}
	.\]
	To say that a vector x is real values with D dimensions, we write $x \in \R^D$ 
	E.g: $x \in \R^2, y \in \R^3 $ 
	Can write column vectors more compactly using parentheses $x = (13)$ 
	The elements of a vector are usually denoted using subscripts
	E.g: if $x = (1 4 5)$ then $x_1 = 1, x_2 = 4, x+3 = 5$
	The transpose of a column vector is a row vector
	\[
		x = \begin{pmatrix} 1\\ 2\\ 3 \end{pmatrix}
		x^T = [1 2 3]
	.\]
	This gives us another way to write column vectors compactly:
	$x = [x_1 x_2 x_3]^T \in \R^3$
	\subsection{Adding and Scaling Vectors}
	To add two row or column vectors of the same dimension, just add their components
	\[
		x = \begin{pmatrix} x_1\\ x_2 \\ \vdots\\ x_D \end{pmatrix} 
		y = \begin{pmatrix} y_1\\ y_2 \\ \vdots\\ y_D \end{pmatrix}
		x+y = \begin{pmatrix} x_1+y_1 \\ x_2 +y_2 \\ \vdots\\ x_D+y_D \end{pmatrix}
	.\]
	You cannot add a row vector to a column vector or vice versa (unless the dimension is 1)
	You cannot add vectors of different dimensions.
	To multiply a vector $x \in \R^D$ by a scalar $\alpha \in \R$, just multiply each component.
	\[
		\alpha x = [\alpha x_1 \alpha x_2 \ldots \alpha x_D]^T
	.\]
	\subsection{Dot product}
	The dot product between two vectors $x,y \in \R^D$ is computed by multiplying the
	corresponding components of x and y and adding up the products.
	\[
	x\cdot y = x_1 y_1 + x_2 y_2 + \ldots = \sum_{i=1}^{N} x_i y_i
	.\] 
	The dot product is also called the inner product or scalar product.
	Alternative notations:
	 \begin{itemize}
		 \item $x^T y$ (dot product as a a matrix multiplication)
		 \item $(x,y)$ or $(x|y)$ (bracket notation, common in physics)
		 \item $xy$ implicit notation
	\end{itemize}
	\subsection{Norms}
	Most common norm is the Euclidean norm or $L_2$ norm, denoted $\mid\mid x\mid\mid_2$
	or just $ \mid  \mid x \mid  \mid $.
	Gives the length of a vector.
	\[
	\mid  \mid x \mid  \mid = \sqrt{x^T x} = \sqrt{\sum_{i}x_i^2} 
	.\] 
	The squared Euclidean norm $ \mid  \mid x \mid  \mid^2$ is often useful:
	\[
	 \mid  \mid x \mid  \mid^2 = x^Tx = x\cdot x
	.\] 
	Another common norm is the $L_1$ norm, which is the sum of absolute values of x
	\[
		\mid  \mid x \mid  \mid_1 = \sum_{i} \mid x_i \mid 
	.\] 
	\subsection{Properties of Norms}
	A norm is any function p from vectors to $\R$ that satisfies:
	\begin{enumerate}
		\item $p(\alpha x) =  \mid \alpha  \mid p(x)$ for $\alpha \in \R$ 
		\item $p(x+y) \le p(x) + p(y)$ (Triangle inequality)
		\item $p(x) = 0 \iff x = 0$
	\end{enumerate}
	We also have $p(-x) = p(x)$
	E.g: $ \mid  \mid 5x \mid  \mid = 5  \mid \mid x  \mid \mid$ for any norm $ \mid  \mid \cdot  \mid  \mid $
	\subsection{Norms and Distance Metrics}
	Any vector space V and norm p can be used to induce a distance metric (define a metric
	space) by defining the distance metric to be $d(x,y) = p(x,y)$ 
	E.g: The space of D dimensional real valued vectors $\R^D$ and the $L_2$ norm give the 
	Euclidean space with metric $d(x,y) =  \mid  \mid x - y  \mid  \mid_2$
	In 2D this gives the familiar Euclidean distance between two points x and y:
	\[
		d(x,y) =  \mid  \mid x-y \mid  \mid_2 = \sqrt{(x_1-y_1)^2+(x_2-y_2)^2} 
	.\] 
	\subsection{Matrices}
\end{document}
