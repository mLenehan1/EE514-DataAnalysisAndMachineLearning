With the explosion of information which has come with the prevalence of social
media, there has also come an explosion of misinformation. Machine Learning
techniques allow the opportunity to automatically determine whether the
information being served up is genuine, or satirical, or in simpler terms
``real'', or ``fake'' news.

\par With social media being one of the largest sources of shared misinformation, the
ability to flag suspect articles could lead to a large decrease in the amount of
misinformation in circulation. However, the classification of these articles
would need to be robust enough to avoid misclassifying, and mislabelling,
articles as ``fake'' or malicious.

\par Using a recently assembled dataset of satirical and non-satirical
headlines, a model can be trained in order to satisfy these requirements. This
assignment aims to investigate the implementation of such a model, and to
examine the accuracy which can be achieved.
