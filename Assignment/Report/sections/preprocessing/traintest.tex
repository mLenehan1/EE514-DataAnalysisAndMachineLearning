\subsection{Train and Test Split}

In order to accurately train and test a classification model, the dataset must
be divided into a training set, and a test set. This removes the possibility
that the model will ``memorize'' the entire dataset, giving a 100\% accuracy.
While performing this split, an unbiased training and test set must be created.
For example, training a model off entirely off ``sarcastic'' items would not
provide good accuracy during the testing phase.

\begin{lstlisting}[language=Python, caption={sklearn ``train\_test\_split''
function},
label={lst:ttsplit}]
X_train, X_test, y_train, y_test = train_test_split(
x, y , shuffle=True, test_size=0.25, random_state=1)
\end{lstlisting}

The scikit-learn python library contains a ``train\_test\_split'' function,
which, as shown in Listing \ref{lst:ttsplit}, can be used to split the feature
and target sets in the proportions given by the ``test\_size'' parameter. The
``shuffle'' parameter allows for the shuffling of the data, which will allow for
less biased split. The ``random\_state'' parameter sets the seed used by the
random number generator, allowing for repeatability within the
testing\cite{sk_ttsplit}.

The function call in Listing \ref{lst:ttsplit} creates a train test split of
75\% to 25\%, which is shuffled to remove bias, with a random state of 1 for
repeatability between tests.

\begin{lstlisting}[language=Python, caption={``countRealFake'' function},
label={lst:realFake}]
def countRealFake(targetTrainSet, targetTestSet):
    print("Real Headlines in Training Set: %s" % sum(targetTrainSet == 0))
    print("Fake Headlines in Training Set: %s" % sum(targetTrainSet == 1))
    print("Real Headlines in Test Set: %s" % sum(targetTestSet == 0))
    print("Fake Headlines in Test Set: %s" % sum(targetTestSet == 1))
\end{lstlisting}

In order to count the number of real and fake headlines within each set, the
``countRealFake'' function is used. This function takes two datasets, namely the
target training set, and the target test set, and prints the total number of
fake and real headlines in each, represented by a ``1'' or a ``0'' respectively.

\begin{lstlisting}[language=Python, caption={``exportTestTrain'' function},
label={lst:exporttt}]
def exportTestTrain(X_train, X_test, y_train, y_test, printMsgs=False):
	Export = X_train.to_json(r'./XTrain.json', orient='records')
	Export = X_test.to_json(r'./XTest.json', orient='records')
	Export = y_train.to_json(r'./YTrain.json', orient='records')
	Export = y_test.to_json(r'./YTest.json', orient='records')
	if(printMsgs == True):
		print("Training Features output to XTrain.json")
		print("Test Features output to XTest.json")
		print("Training Targets output to YTrain.json")
		print("Test Targets output to YTest.json")
\end{lstlisting}

In order to export the feature and target training and test sets, the
``exportTestTrain'' function is used. As shown in Listing \ref{lst:exporttt},
this function uses the pandas library ``to\_json'' function to output the json
object to a file. The ``orient'' parameter is set to ``records'' in order to
make the file more human readable.
