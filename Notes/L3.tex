\documentclass[a4paper]{article}

\usepackage[utf8]{inputenc}
\usepackage[T1]{fontenc}
\usepackage{textcomp}
\usepackage[dutch]{babel}
\usepackage{amsmath, amssymb}


% figure support
\usepackage{import}
\usepackage{xifthen}
\pdfminorversion=7
\usepackage{pdfpages}
\usepackage{transparent}
\newcommand{\incfig}[1]{%
	\def\svgwidth{\columnwidth}
	\import{./figures/}{#1.pdf_tex}
}

\pdfsuppresswarningpagegroup=1

\begin{document}
\section{Data Summarization}
\subsection{Distributions}
The data distrib describes prob. of random value taking a particular value
\par E.g: IQ follows Normal (Gaussian) distrib.
\par Random no. generator generates random nos. that follow uniform distribution
in [0,1]
\par A roll of a biased dice has a categorical distrib. over the values
$\{1,2,3,4,5,6\} $
 \subsection{Measures of central tendency}
 Sample statistic is a measurement on a sample from a distrib. calculated by
 applying a function to the sample
 \par Measures of central tendency are sample stats that attempt to capture
 where middle of distrib is.
 \par Three common ways of measureing central tendency
 \begin{itemize}
 	\item Mean
	\item Median
	\item Mode
 \end{itemize}
 \subsection{Mean, Median, Mode}
 \begin{itemize}
 	\item Arithmetic Mean:
	\begin{itemize}
		\item Sum of observations divided by no. of observations
	\end{itemize}
 \end{itemize}
 \[
 \overline{x} = \frac{1}{N}\sum_{n=1}^{N} x_n
 .\]
 \begin{itemize}
 	\item Median:
	\begin{itemize}
		\item Middle value that separates higher and lower half of
			dataset
		\item Sort data and take mid value
		\item Even number of observations: take arithmetic mean of
			middle two values
		\item Robust Statistic (less sensitive to outliers)
	\end{itemize}
	\item Mode:
	\begin{itemize}
		\item Most freq. value in dataset
		\item Suitable measure of central tendency for nominal variables
		\item Easy to compute for discrete values
		\item Not straightforward for continuous distrib.
	\end{itemize}
 \end{itemize}
\begin{table}[htpb]
	\centering
	\begin{tabular}{c c c c}
		& Dimension & Type & Calculate \\
		Mean & N & Quantitative & np.mean \\
		Median & 1 & Ordinal & np.median \\
		Mode & N & Nominal & From histogram
	\end{tabular}
\end{table}
\subsection{Sample Mean vs. Population Mean}
Population mean is trye mnean for entire pop.: $\mu$
\par Sample mean is calculated mean from sample of pop: $\overline{x}$
\par E.g. pop. mean human height would require measuring height of every person
on earth. Sample mean 100 randomly chosen people
\par As sample size increases, sample mean approaches pop. mean
\[
\lim_{n \to \infty} \overline{x}_n = \mu
.\]
\subsection{Measures of Statistical Dispersion}
Measure how stretched/squeezed a distrib is
\par Most common methods:
\begin{itemize}
	\item Variance $\sigma^2$
	\item Std. Dev $\sigma$
	\item Interquartile Range (IQR)
\end{itemize}
Can be defined for sample or pop.
\subsection{Variance, SD}
\begin{itemize}
	\item Variance
	\begin{itemize}
		\item Expected squared dev from mean
	\end{itemize}
\end{itemize}
\[
	\delta^2=E[(X-EX)^2]
.\]
\[
	=E[(X-\mu)^2]
.\]
\[
	\hat{\sigma}^2 \frac{1}{N}\sum_{i=1}^{N} (x_i-\overline{x})^2
.\]
Usually don't know pop mean $\mu$, so we just use sample mean:
\[
	\hat{\sigma}_N^2=\frac{1}{N}\sum_{i=1}^{N} (x_i-\overline{x})^2
.\]
For small samples, this is biased, use bias corrected estimator:
\[
	\hat{\sigma}_{N-1}^2=\frac{1}{N-1}\sum_{i=1}^{N} (x_i-\overline{x})^2
.\]
Above known as Bessel's correction
\par Std. dev is sqrt of (possibly bias corrected) variance
\[
\delta=\sqrt{\sigma^2}
.\]
\subsection{Range and interquartile range}
Range of sample is diff. between min and max values
\par Range is not robust stat
\par IQR is range in which $50\%$ of data lies
\[
IQR=Q_3 - Q_1
.\]
\subsubsection{Interquartile Range}
IQM can be used on ordinal vars and often makes more sense than the std. dev
\subsection{Sufficient Statistics}
Stat. is sufficient wrt stat model and assoc. unknown paoram when no other stat
that can be calculated from the same sample provides any additional info as tp
the val of the param
\par For Normal distribution, sufficient stats are mean and std. dev
\[
\null =\{\mu, \sigma\}
.\]
Once sufficient stats. known you know everything there is to know about distrib
of var.
\subsection{Skewness and Kurtosis}
Most data not normally distrib.
\par Skewness: measure of assymmetry of prob. distrib. of real values random var
about its mean
\par Kurtosis: Measure of how heavy tails are of prob. distrib of real-valued
random var
\begin{itemize}
	\item Pearsons moment coefficient of skewness:
	\begin{itemize}
		\item Third standardized moment:
	\end{itemize}
\end{itemize}
\[
	_1 = E[(\frac{X-\mu}{\sigma \\})^3]
.\]
\begin{itemize}
	\item Pearson's moment coefficient of kurtosis:
	\begin{itemize}
		\item Fourth standardized moment:
	\end{itemize}
\end{itemize}
\[
	\Beta_2 = E[\frac{X-\mu}{\sigma)^{4}}]
.\]
\begin{itemize}
	\item Replace mu and sigma with sample stat.s to calculate for sample
\end{itemize}
\subsection{Conditioning on Categorical Variable}
Conpute stats based on subsets of sample
\par Subsets selected by grouping on categorical vars
\par These are stats on the conditions distrib.
\subsection{Statistics of association between attributes}
Covariance: how two vars vary together
\[
	cov(X,Y)=E[(X-\mu_x)(Y-x_y)]
.\]
Pearson product-moment correlation coefficient (Pearson's rho):
\begin{itemize}
	\item Measure of linear relationship between quantitative vars
	\item Normalized version of the covariance
\end{itemize}
\[
	\rho_{X,Y}=\frac{cov(X,Y)}{\sigma_X\sigma_Y}
.\]
\begin{itemize}
	\item Ranges from -1 to 1
	\item >0: + correlation
	\item <0: - correlation
	\item +1 indicates perfect linear relationshiop between X and Y
	\item -1 is perfect negative correlation
\end{itemize}
Not suitable for ordinal variables
\subsection{Spearman Correlation}
Spearmans rank correlation coefficient (Spearman's rho):
\begin{itemize}
	\item Pearson correlation between rank values of those two vars
\end{itemize}
\par Pearson's correlation assesses linear relationships, Spearmans correlation
assess monotonic relationships (whether linear or not)
\section{Data Visualization}
\subsection{Why Visualize Data?}
\begin{itemize}
	\item Explore
	\begin{itemize}
		\item Understand data and relationships between attribs.
		\item Generate new hypothese
		\item Detect errors and anomalies
	\end{itemize}
	\item Confirm
	\begin{itemize}
		\item Hypotheses already generated
		\item Check assumptions
		\item Verify conclusions
	\end{itemize}
	\item Communicate
	\begin{itemize}
		\item Analysis has been completed
		\item Conclusions verified
		\item Task: Present and Communicate results
		\item Inform, persuade, educate, entertain
	\end{itemize}
\end{itemize}
\subsection{Statistical Data Visualization}
Provide a set of standard visual tools for:
\begin{itemize}
	\item Undestanding the shape and distrib. of data
	\item Understanding relationships between attribs.
	\item Understanding composition and part-whole relationships
	\item Comparing quantities
\end{itemize}
\end{document}
